Now we have to configure the \texttt{nginx} server in order to expose our .NET Core web application to the outside.
As a result of this section web app will be exposed and accessible via VM's external IP address.
Let's install it using the commands
\begin{itemize}
    \item \texttt{sudo apt update -y}
    \item \texttt{sudo apt install -y nginx build-essential}
\end{itemize}
Terminal output:
\begin{figure}[H]
    \centering
    \includegraphics[width=1\textwidth]{img/06_install_nginx}
    ~\caption{Ubuntu install nginx terminal output.}\label{fig:figure15}
\end{figure}
Next, it is necessary to create nginx configuration that exposes our application, that is
\begin{spverbatim}
    server {
        server_name STATIC_IP_ADDRESS_OF_VM;

        location / {
            include proxy_params;
            proxy_pass http://127.0.0.1:8080;
        }

        location /swagger {
            include proxy_params;
            proxy_pass http://127.0.0.1:8080;
        }

        location /api {
            include proxy_params;
            proxy_pass http://127.0.0.1:8080;
        }

        location /notify {
            proxy_pass http://127.0.0.1:8080;
            proxy_http_version 1.1;
            proxy_set_header Upgrade $http_upgrade;
            proxy_set_header Connection "upgrade";
            proxy_set_header Host $host;
            proxy_cache_bypass $http_upgrade;
        }
    }
\end{spverbatim}
We create it at the following path on behalf of our Azure VM via SSH
\begin{center}
    \texttt{sudo vim /etc/nginx/conf.d/back.mangomesenger.company.conf}
\end{center}
Restart nginx and validate its state using the commands
\begin{itemize}
    \item \texttt{sudo systemctl restart nginx}
    \item \texttt{sudo nginx -t}
\end{itemize}
Terminal output:
\begin{figure}[H]
    \centering
    \includegraphics[width=1\textwidth]{img/06_test_nginx}
    ~\caption{Restart and test nginx terminal output.}\label{fig:figure16}
\end{figure}
Now we must be able to find our application listening to the
\begin{center}
    \texttt{http://STATIC\_IP\_ADDRESS\_OF\_THE\_VM}
\end{center}
And actually it works as expected
\begin{figure}[H]
    \centering
    \includegraphics[width=1\textwidth]{img/06_view_in_browser}
    ~\caption{.NET Core web app accessed via browser using static IP address of the virtual machine.}\label{fig:figure17}
\end{figure}
